% !TEX program = xelatex
\documentclass{article}

%% ---- ตั้งค่าให้ตัดคำภาษาไทย ---- %%
\XeTeXlinebreaklocale "th"
\XeTeXlinebreakskip = 0pt plus 0pt % เพิ่มความกว้างเว้นวรรคให้ความยาวแต่ละบรรทัดเท่ากัน


%% ---- font settings ---- %%
\usepackage{fontspec} % For Thai font
\defaultfontfeatures{Mapping=tex-text} % map LaTeX formating, e.g., ``'', to match the current font
% To change the main font, uncomment one of the below command.
% \setmainfont{TeX Gyre Termes} % Free Times
% \setsansfont{TeX Gyre Heros} % Free Helvetica
% \setmonofont{TeX Gyre Cursor} % Free Courier
\newfontfamily{\thaifont}[Scale=MatchUppercase,Mapping=textext]{Laksaman} % ตั้งฟอนต์หลักภาษาไทย
\newenvironment{thailang}{\thaifont}{} % create environment for Thai language
\usepackage[Latin,Thai]{ucharclasses} % ตั้งค่าให้ใช้ "thailang" environment เฉพาะ string ที่เป็น Unicode ภาษาไทย
\setTransitionTo{Thai}{\begin{thailang}}

\setTransitionFrom{Thai}{\end{thailang}}

%% ---- spacing between lines ---- %%
\usepackage{setspace}
% \singlespacing % default setting
% \onehalfspacing % recommend using this for Thai language

%% ---- using alphabatic language ---- %%
\usepackage{polyglossia}
\setdefaultlanguage{english} % it is preferrable to set English as the main language, since the numeric system is compatible with most LaTeX features such as 'enumerate' and so on
\setotherlanguages{thai}

\AtBeginDocument\captionsthai % allow captions to be in Thai


%% ---- hyperref settings ---- %%
\usepackage{hyperref}
\usepackage{url}
\usepackage{cite}
\usepackage{xcolor}
\hypersetup{
    colorlinks,
    linkcolor={red!50!black},
    citecolor={blue!50!black},
    urlcolor={blue!80!black}
    }

%% ---- misc. packages ---- %%
\usepackage{metalogo} % for extended LaTeX logo such as XeTeX

%% ==== Title, Auth, Date ==== %%
\title{Thai \XeTeX ~Template \thanks{Thanks \href{https://github.com/mathmd/polygloTeX}{mathmd's Github repo} for nice polyglot template.}}
\author{Kittipos Sirivongrungson }
\date{January 2021}

%%%%%%%%%%%%%%%% Begin Document %%%%%%%%%%%%%%%% 
\begin{document}
\sloppy % ช่วยตัดคำภาษาไทย
\maketitle
\begin{abstract}
	This document use \texttt{fontspec}, \texttt{polyglossia} and \texttt{ucharclasses} packages to provide typesetting in Thai.
	ฟอนต์ภาษาไทย \textenglish{\texttt{Laksaman}} เป็นของโครงการ \textenglish{TLWG (Thai Linux Working Group)} ที่มีความคล้ายกับ \textenglish{\texttt{TH SarabunPSK}} ที่ใช้สำหรับการจัดทำเอกสารใน MS Word	
\end{abstract}

\section{Alphabatic languages}
Alphabatic scripts, e.g., Thai, Devanagari, Arabic, Cyrillic, etc., can be used simultaneously by using \texttt{polyglossia} package.
To ease writing experience, \texttt{ucharclasses} package can be used to automate the language environment switching while the main language remains English.


\subsection{ภาษาไทย}
\subsubsection{With onehalf spacing}

\begin{onehalfspace}
	เนื่องจากว่าใช้แพ็คเกจ \texttt{ucharclasses} เราจึงไม่จำเป็นต้องเรียก environment ภาษาไทยทุกครั้ง ทำให้สามารถใช้หลายภาษาพร้อมกันได้โดยสะดวก
	ในย่อหน้าที่\textbf{ภาษาไทย}เป็นหลักควรตั้งระยะห่างระหว่างบรรทัดด้วยคำสั่ง \texttt{\textbackslash onehalfspacing} หรือเรียก environment ชื่อ \texttt{onehalfspace}
\end{onehalfspace}

\subsubsection{Without onehalf spacing}
    เนื่องจากว่าใช้แพ็คเกจ \texttt{ucharclasses} เราจึงไม่จำเป็นต้องเรียก environment ภาษาไทยทุกครั้ง ทำให้สามารถใช้หลายภาษาพร้อมกันได้โดยสะดวก
    ในย่อหน้าที่\textbf{ภาษาไทย}เป็นหลักควรตั้งระยะห่างระหว่างบรรทัดด้วยคำสั่ง \texttt{\textbackslash onehalfspacing} หรือเรียก environment ชื่อ \texttt{onehalfspace}




\end{document}